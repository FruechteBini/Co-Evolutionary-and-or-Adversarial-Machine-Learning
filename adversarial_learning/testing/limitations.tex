As mentioned before, a thorough testing process is necessary before integrating the learning system in a safety critical domain. When using a traditional random testing algorithm, usually the confidence that the failure rate of a policy is below $\epsilon$ requires at least $1/\epsilon$ testing episodes. This might be inefficient and time-intensive. For example in a autonomous driving domain, a car that crashes once in 100 million miles would not be brought on the market. More accurately, if there is a greater failure probability than $\epsilon = 10^{-8}$ per mile. In order to achieve reasonable confidence, the manufacturer would need to test-drive the car for at least $10^{-8}$ miles, which may be prohibitively expensive \cite{uesato18}. To overcome these obstacles, a novel testing approach is needed.
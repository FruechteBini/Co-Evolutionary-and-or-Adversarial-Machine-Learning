In recent years, reinforcement learning has driven impressive advances in machine learning. Despite revolutionary progress, limitations still exists due to memory complexity, computational complexity and sample complexity. To enhance reinforcement learning algorithms, biologically influenced extensions of the basic concepts have emerged from recent studies. In biology, it is often observed that antagonistic populations are trying to outperform each other. Competitive co-evolution results in fast improvements, increased genetic robustness and the ability to survive in more complex environments. This observation can be transferred to reinforcement learning and has been proven to outperform conventional concepts. In this work, we present and compare different adversarial and competitive co-evolutionary approaches for training and testing reinforcement learning agents.